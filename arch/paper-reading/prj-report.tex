\documentclass[a4paper,12pt]{article}
\usepackage[top=1in,bottom=1in,left=1.25in,right=1.25in]{geometry}
\usepackage{indentfirst}
\usepackage{titlesec}
\usepackage{latexsym}
\usepackage{amsmath}
\usepackage{amssymb}
\usepackage{CJK}

\setlength{\parindent}{2em}

\begin{document}
\begin{CJK*}{UTF8}{song}
	
	\newcommand{\chuhao}{\fontsize{42pt}{\baselineskip}\selectfont}
	\newcommand{\xiaochuhao}{\fontsize{36pt}{\baselineskip}\selectfont}
	\newcommand{\yihao}{\fontsize{28pt}{\baselineskip}\selectfont}
	\newcommand{\erhao}{\fontsize{21pt}{\baselineskip}\selectfont}
	\newcommand{\xiaoerhao}{\fontsize{18pt}{\baselineskip}\selectfont}
	\newcommand{\sanhao}{\fontsize{15.75pt}{\baselineskip}\selectfont}
	\newcommand{\sihao}{\fontsize{14pt}{\baselineskip}\selectfont}
	\newcommand{\xiaosihao}{\fontsize{12pt}{1.3\baselineskip}\selectfont}
	\newcommand{\wuhao}{\fontsize{10.5pt}{1.3\baselineskip}\selectfont}
	\newcommand{\xiaowuhao}{\fontsize{9pt}{\baselineskip}\selectfont}
	\newcommand{\liuhao}{\fontsize{7.875pt}{\baselineskip}\selectfont}
	\newcommand{\qihao}{\fontsize{5.25pt}{\baselineskip}\selectfont}

	\title{{\Large{\CJKfamily{hei}《Uses and Abuses of Amdahl's Law》阅读报告}}}
	\author{张涛 \\ 00648331\\zhtlancer[at]gmail.com}
	\date{June 2009}
	\maketitle

	\section{\large{论文内容概述\cite{bib1}}}
	\xiaosihao这篇论文主要通过对平时的实际应用领域中,Amdahl定律的若干应用方式以及一些错误的使用方式进行讨论研究,进而总结得出一些指出Amdahl定律不合理的观点的错误性。同时,通过正误方法的对比,可以加深对于Amdahl定律的理解,并帮助更好地在实践中使用它。

	其中,论文中分别从{\CJKfamily{hei}多进程并行计算(Multiprocessing)}、{\CJKfamily{hei}存储器层次结构设计(Memory hierarchy design)}以及{\CJKfamily{hei}指令系统和处理器设计(Instruction set and processor design)}等领域,对于Amdahl定律的应用方法进行了阐述,并列举出若干人们平时应用Amdahl定律过程中的误区,分析人们在应用时应该注意的问题。分别总结如下。

		\subsection{多进程并行计算领域}
		\xiaosihao很多人在使用Amdahl定律来估计某些应用在并行计算中所能获得的性能提升时,与实际情况存在较大误差。Gustafson在此背景下提出了Gustafson定律:\begin{displaymath}S=\frac{1}{1+(1-g)*N/g}\end{displaymath}而作者结合Yuan Shi的研究结果,说明了Gustafson定律与Amdahl定律从本质上来讲是一致的。而人们在使用Amdahl定律估算性能提升比时的误差主要来源于错误地使用上述公式中的参数g作为Amdahl定律中的参数f。

		另外,在某些并行计算应用中,性能提升比甚至会达到超线性的程度,这也是与Amdahl定律相违背的。作者以一个排序问题为例,指出其本质原因在于这些应用在并行计算过程中采用了不同的算法,这本来不符合Amdahl定律的前提条件,所以用Amdahl定律来估计它的性能提升比是不合理的。

		\subsection{存储器层次结构设计领域}
		\linespread{1.1}此部分通过将存储器层次结构的性能提升比公式以Amdahl定律形式表示如下\begin{displaymath}S=\frac{N'}{h+(1-h)*N'},\ (N'=T_{memory}/T_{cache},\ h:cache\ hit\ ratio)\end{displaymath},由此公式可以看出提升Cache性能对于提升整个存储系统性能的作用的局限性。同时也说明Amdahl定律适用范围很广泛。

		\subsection{指令系统与处理器设计领域}
		\linespread{1.1}Amdahl定律在此领域中的应用是本学期体系结构课程学习中的一部分。即在指令系统设计、流水线设计以及处理器实现过程中,通过提升单元性能对于整个实现的性能影响受制于该单元在整体执行中所占比例。这一点在权衡设计/实现时的性能提升代价比时非常有用,可以避免不必要的优化所带来的开销。

	\section{\large{个人收获与体会}}
	\linespread{1.1}此次参与课程项目开发算是第一次开源项目开发经历。总体来说,收获颇多,而且对于一些规模比较大的开源项目开发模式有了更加深入的了解。并且对于本学期学习的一些关于开源开发的工具使用也有了实际使用经验,掌握更加深入。以下分别进行说明。

	\begin{thebibliography}{99}

		\bibitem{bib1}
			S. Krishnaprasad:"Uses and Abuses of Amdahl's Law" in JCSC 17,2(Dec. 2001)
	\end{thebibliography}

\end{CJK*}
\end{document}

\documentclass[a4paper,12pt]{article}
\usepackage[top=1in,bottom=1in,left=1.25in,right=1.25in]{geometry}
\usepackage{indentfirst}
\usepackage{titlesec}
\usepackage{latexsym}
\usepackage{amsmath}
\usepackage{amssymb}
\usepackage{CJK}

\setlength{\parindent}{2em}

\begin{document}
\begin{CJK*}{UTF8}{song}
	
	\newcommand{\chuhao}{\fontsize{42pt}{\baselineskip}\selectfont}
	\newcommand{\xiaochuhao}{\fontsize{36pt}{\baselineskip}\selectfont}
	\newcommand{\yihao}{\fontsize{28pt}{\baselineskip}\selectfont}
	\newcommand{\erhao}{\fontsize{21pt}{\baselineskip}\selectfont}
	\newcommand{\xiaoerhao}{\fontsize{18pt}{\baselineskip}\selectfont}
	\newcommand{\sanhao}{\fontsize{15.75pt}{\baselineskip}\selectfont}
	\newcommand{\sihao}{\fontsize{14pt}{\baselineskip}\selectfont}
	\newcommand{\xiaosihao}{\fontsize{12pt}{1.3\baselineskip}\selectfont}
	\newcommand{\wuhao}{\fontsize{10.5pt}{1.3\baselineskip}\selectfont}
	\newcommand{\xiaowuhao}{\fontsize{9pt}{\baselineskip}\selectfont}
	\newcommand{\liuhao}{\fontsize{7.875pt}{\baselineskip}\selectfont}
	\newcommand{\qihao}{\fontsize{5.25pt}{\baselineskip}\selectfont}

	\title{{\Large{\CJKfamily{hei}《Uses and Abuses of Amdahl's Law》阅读报告}}}
	\author{张涛 \\ 00648331\\zhtlancer[at]gmail.com}
	\date{June 2009}
	\maketitle

	\section{\large{论文内容概述}}
	\xiaosihao这篇论文主要通过对平时的实际应用领域中,Amdahl定律的若干应用方式以及一些错误的使用方式进行讨论研究,进而总结得出一些指出Amdahl定律不合理的观点的错误性。同时,通过正误方法的对比,可以加深对于Amdahl定律的理解,并帮助更好地在实践中使用它。

	其中,论文中分别从{\CJKfamily{hei}多进程并行计算(Multiprocessing)}、{\CJKfamily{hei}层次存储结构设计(Memory hierarchy design)}以及{\CJKfamily{hei}指令系统和处理器设计(Instruction set and processor design)}等领域,对于Amdahl定律的应用方法进行了阐述,并列举出若干人们平时应用Amdahl定律过程中的误区,分析人们在应用时应该注意的问题。分别总结如下。

	\begin{itemize}
		\item{{\CJKfamily{hei}多进程并行计算领域}\\很多人在使用Amdahl定律来估计某些应用在并行计算中所能获得的性能提升时,与实际情况存在较大误差。Gustafson在此背景下提出了Gustafson定律:\begin{displaymath}S=\frac{1}{1+(1-g)*N/g}\end{displaymath}而作者结合Yuan Shi的研究结果,说明了Gustafson定律与Amdahl定律从本质上来讲是一致的。而人们在使用Amdahl定律估算性能提升比时的误差主要来源于错误地使用上述公式中的参数g作为Amdahl定律中的参数f。}
		\item{{\CJKfamily{hei}书签/附注管理部分}}
			\begin{itemize}
				\item{数据组织形式:\wuhao采用双向链表,顺序与书签/附注在文档中的位置顺序一致,提供双向移动以及插入删除操作}
				\item{外部文件存储方式:\wuhao使用轻量级的嵌入式数据库sqlite3进行书签/附注存储,分别保存书签/附注所在页面编号、页内偏移以及文字内容。存储的数据库文件名及路径由用户界面指定,当不存在时创建新的数据库文件。}
				\item{TODO:\wuhao对书签/附注的组织也实现缓存方式,以提高书签/附注访问效率。这里需要注意书签/附注是可读写的,所以需要注意更新时保持一致性的问题。}
			\end{itemize}
	\end{itemize}

	\section{\large{个人收获与教训}}
	此次参与课程项目开发算是第一次开源项目开发经历。总体来说,收获颇多,而且对于一些规模比较大的开源项目开发模式有了更加深入的了解。并且对于本学期学习的一些关于开源开发的工具使用也有了实际使用经验,掌握更加深入。以下分别进行说明。

	\begin{enumerate}
		\item{利用GCC+automake+autoconf开发工具进行项目开发。\\
			从最初使用Linux已经有将近三年,刚开始的时候也会用GCC编译一些比较小的小程序(如POJ算法题目),大多都是单源文件单模块,且不依赖其它第三方库,所以基本都是直接用GCC手动编译,不需要使用automake/autoconf等自动build工具。\\
			而这次项目开发,算是第一次真正进行较大的项目开发,而且利用了sqlite3这个开源的外部库支持。在开发过程中,才发现这些自动开发工具的重要性。其中主要是automake/autoconf对于源文件编译/链接依赖以及外部链接库的链接支持。}
		\item{利用SVN版本控制工具进行项目代码管理以及合作开发。\\
			在以前的项目开发过程中,一般都采取多人分别开发再合并源代码的方式。当源代码划分比较合理的时候也许没太大问题。但当源代码划分不合理,或代码交叉较多时,这种开发方式就显得非常不方便,无法满足多人同时开发的需要。而且在开发过程中,有时会不可避免的在代码中引入新的错误,导致源代码作废,所以开发者需要不时对当前有效代码进行手动备份,非常麻烦。另外由于每个人对项目的更改无法及时在其他开发者的代码中应用,所以效率很低。\\
			而本次开发中,第一次利用SVN进行项目合作开发,避免的上述问题,并且为我们提供了一个非常方便的历史代码备份方法,方便进行代码回退等。同时也便于对项目感兴趣的人对代码进行浏览。}
		\item{对现有第三方开源资源的利用。\\
			在我负责的书签/附注管理模块中,使用了一个非常优秀的第三方开源项目资源sqlite3,以为项目提供一个不依赖于使用者环境的数据库支持。第三方的开源资源极大提高了项目的开发效率,并且保证了项目的质量。}
		\item{关于本次项目开发中的教训。\\
			项目开发过程中,大家都投入了很大精力,对最终项目成果比较期待。但是由于项目最初接口定义方面的模糊性,导致最终项目模块合并失败。总结下大致有如下几个教训:
			\begin{itemize}
				\item{项目第一个里程碑目标太高,导致达到第一个里程碑经历较长时间,而且开发量较大,使得不同模块的分歧越来越大。如果最初里程碑目标能更实际一些,开发难度小一些,也许处理接口冲突更加方便。}
				\item{没有遵循"增量开发"的理念。如上条所述,我们的首个可用代码没有产生就夭折了。如果要遵循"增量开发"的方式,我们应该先实现一个功能较简单的可用程序,然后在此基础上不断增加新的功能,不断完善。这样也更便于项目编译和调试。在"增量开发"过程中,需要注意一点的是需要充分考虑对于可扩展性的设计,否则后期对于项目的功能扩充难以进行}
				\item{没有充分利用好提供的开发工具。课程为我们提供的trac版本跟踪工具,以便于我们项目开发过程中开发者之间相互交流和使用者对于项目bug提交,以及项目发布。但是我们没有及时发现这一强大工具的作用,这对我们项目最终的失败也有一定关系。}
			\end{itemize}
			}
	\end{enumerate}

	\begin{thebibliography}{99}

		\bibitem{bib1}
			S. Krishnaprasad:"Uses and Abuses of Amdahl's Law" in JCSC 17,2(Dec. 2001)
	\end{thebibliography}

\end{CJK*}
\end{document}
